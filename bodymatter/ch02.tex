\chapter[Installation]{不怕命令行 }

无论是安装还是使用 \ConTeXt,皆需要对命令行环境有所了解。原本未有介绍这方面知识的计划,但是考虑到我正在写一份世上最好的 \ConTeXt\ 入门文档,便有了些许动力。本章先分别介绍 Windows、Linux 和 macOS 系统的命令行环境的基本用法,以刚好满足安装和运行 \ConTeXt\ 的需求为要。倘若对命令行环境已颇为熟悉,可直接阅读 \in[installation] 和 \in[ctx-in-texlive] 节。

\section{任务}

使用命令行环境,在文件系统中,创建一个名为 foo 的目录,在该目录内创建一份 Shell 脚本,令其可在命令行窗口中输出「不怕命令行」,执行该脚本,查看其输出。

\section{Windows 命令行}

Windows 用户似乎畏惧甚至厌憎命令行环境,甚至很多人认为命令行环境是早已被淘汰的上个世纪的产物,因此要教会他们如何使用命令行环境,通常会有些麻烦,我当勉力为之。

在 Windows 系统中打开一个命令行窗口,有很多种方法,其中最快的应当是使用如图 \in[win-r] 所示快捷键「Win + R」,打开「运行」对话框,在其中输入「cmd」后点击「确定」按钮或单击「Enter」键,即可打开与图 \in[cmd-window] 类似的命令行窗口。
