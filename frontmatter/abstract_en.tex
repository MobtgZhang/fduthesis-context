\startcomponent abstract_en
\product fduthesis

\title{Abstract}

\indentation

ConTeXt is a powerful typesetting system designed for users who require fine-grained control over layout and structure, making it particularly suitable for multilingual documents, technical manuals, and academic publishing.
It is built on the LuaTeX engine and supports modular design, style reuse, and dynamic content generation.
Compared to LaTeX, ConTeXt offers a more consistent syntax and more flexible style definitions, allowing users to customize page layout, font settings, section formatting, and image-text integration using commands such as `\type{\startsection}`, XXXX XXXX XXXXXX XXXXX XXXXXXX XXXXXX `\type{\definefontfamily}`, and `\type{\setuphead}`.
It includes built-in support for CJK scripts like Chinese (via the `zhfonts` module or custom font settings), and supports PDF output, graphical illustration (via MetaFun), and automation tasks such as cross-referencing, table of contents, and indexing.
ConTeXt is operated primarily through the command line (e.g., using `context` for compilation and `mtxrun` for management), and is well-suited for users seeking high-quality typesetting within a unified framework.

\stopcomponent